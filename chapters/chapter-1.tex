\documentclass[../main.tex]{subfiles}

\begin{document}
The present work is the result of an internship in the Interactive Augmented Modelling
(IAM) at the Computer Vision Center (CVC). It is part of an ongoing collaboration,
TOPiomics, between the IAM and the Radiomics Group at the Vall d'Hebron Institute of
Oncology (VHIO).  

This chapter provides some background and context for the state of the study and the
field of radiomics more broadly, as well as a description of how the results of this
thesis fit into the bigger picture. 

\section{An overview of radiomics}
Radiomics is a relatively recent medical field. Its principal aim is to establish a
relationship between a patient's response to treatment and what are known as
\emph{radiomic features} of a lesion. These are parameters systematically extracted from
scans of a lesion such as a tumour or a nodule. The hypothesis is that there is information
that can be extracted from this more systematic study of medical scans, which is not
accessible through the more traditional qualitative observation by a medical professional.
Furthermore, since imaging is much less invasive than other techniques such as biopsy, if
reliable, radiomics could become a useful way of monitoring a patient's evolution and
response. 

\subsection{Acquisition of radiomic data}
The process of extracting radiomic features begins with the actual imaging of the lesion
or tumour: a series of 2D sectional scans are obtained, from which a 3D reconstruction is
assembled. It is then the work of a medical professional to manually delimit the contours
of the actual lesion, which is known as the \emph{region of interest} or ROI. This process is
called \emph{segmentation}. Once this is done, the actual work of computing the radiomic
features takes place. For example, one can gather statistical descriptors of the
distribution of brightness of the pixels in the ROI, or calculate geometric parameters of
the mesh defined by the boundary of the ROI such as volume, surface area or sphericity.
There is a Python package, \textsf{PyRadiomics}, built specifically for the task of
extracting radiomic features. \cite{pyradiomics} describes the whole list of parameters it
can compute. The collection of these features is called the \emph{radiomic signature}. 

It is not obvious which of these features will best correlate with the response to
treatment. Even worse, some of them might be highly dependent on the segmentaion of the
ROI, so that if a different professional performs segments the ROI with slight variations,
the value of the features changes wildly. It is desirable to select those featues which,
first, are stable under variations in segmentation, called \emph{reproducible}, and are
best correlated to the response to treatment. 

\subsection{The nature of radiomic data}
The aspect of radiomics that is more interesting to a mathematician is the work of
attempting to extract patterns from the radiomic data. However, the data sets available
have two significant problems: on the one hand, since all of these features are extracted
from clinical cases, the size of available data is necessarily limited, and on the other
hand the dimension of the data size can be very high since in the acquisition stage as
many features as possible are computed. This means that traditional techniques from
statistics and big data analysis which rely on large volumes of data are not well suited
to radiomics.

\section{Prior work} \label{sec:prior work}
In view of the particular needs of radiomics, the TOPiomics project was developed with the
aim of bridging the gap between the needs of medical professionals and the tools
mathematicians have at their disposal. Specifically, as described in \cite{topiomics,
attract}, the project intends to develop a method of \emph{outlier detection}, i.e.
patients with abnormal radiomic signatures. This is important because these abnormalities
might indicate that existing treatments could fail with this patient and thus signal the
need to develop specially tailored treatments. An early detection of these outliers is
crucial for a positive evolution of the patient. 

The idea then is to employ methods from \emph{topological data analysis}, hence the name
TOPiomics, for the detection of outliers. A first attempt described in \cite{outliers}
develops an approach based on the \emph{mutual k-nearest neighbours graph}, which is used
to find clusters in the data and give a measure of ``outlier-ness'' to each patient in the
data set. There is, however, an ammount of parameter fiddling involved in this method (the
\( k \) in \( k \)-nearest neighbours) to guarantee that the results are accurate and not
an artifact of the choice of \( k \). \emph{Persistent homology} seems specially
well-suited to solve this problem since its main application is the analysis of data at
many scales. The work of this thesis then is to develop an outlier detector based on
persistent homology. 

\end{document}
