\documentclass[../main.tex]{subfiles}

\begin{document}
The present work is the result of an internship in the Interactive Augmented Modelling
(IAM) at the Computer Vision Center (CVC). It is part of an ongoing collaboration,
TOPiomics, between the IAM and the Radiomics Group at the Vall d'Hebron Institute of
Oncology (VHIO).  

This chapter provides some background and context for the state of the study and the
field of radiomics more broadly, as well as a description of how the results of this
thesis fit into the bigger picture. 

\section{Prior work}
Radiomics is a relatively recent medical field. Its principal aim is to establish a
relationship between a patient's response to treatment and what are known as
\emph{radiomic features} of a lesion. These are parameters systematically extracted from
scans of a lesion or other feature. The hypothesis is that there is information
that can be extracted from this more systematic study of medical scans, which is not
accessible through the more traditional qualitative observation by a medical professional.
Furthermore, since imaging is much less invasive than other techniques such as biopsy, if
reliable, radiomics could become a useful way of monitoring a patient's evolution and
response. 

However, since all of these features are extracted from clinical cases, the size of
available data is necessarily limited. This means that traditional techniques
from statistics and big data analysis which rely on large volumes of data are not well
suited to radiomics. In this context, TOPiomics \cite{topiomics, attract} attempts to
bridge the gap by employing tools from \emph{topological data analysis}. Specifically, the
goal of TOPiomics is to detect outliers as early as possible, so that specialised
treatments can be developed as soon as possible. The approach thus far is based on what is
known as the \emph{Mutual \( k \)-Nearest Neighbours graph} \cite{outliers}. Nevertheless,
there is an amount of parameter fiddling involved (the \( k \) in \( k \)-Nearest
neighbours)	to guarantee that the results are accurate and not an artifact of the choice
of \( k \). The idea was then to use \emph{persistent homology}, since it is suited to the
study of the topological structure of a data set at various scales. , since it is suited
to the study of the topological structure of a data set at various scales. 

The work performed in this thesis is the implementation of a library which computes
persistent homology and its application to radiomic datasets in an attempt to detect
outliers. 

\section{Radiomic data}
The process of extracting radiomic features begins with the actual imaging of the lesion
or tumour: a series of 2D sectional scans are obtained, from which a 3D reconstruction is
assembled. It is then the work of a medical professional to manually delimit the contours
of the actual lesion, which is known as the \emph{region of interest} or ROI. This process is
called \emph{segmentation}. Once this is done, the actual work of computing the radiomic
features takes place. For example, one can gather statistical descriptors of the
distribution of brightness of the pixels in the ROI, or calculate geometric parameters of
the mesh defined by the boundary of the ROI such as volume, surface area or sphericity.
The Python package, \textsf{PyRadiomics} \cite{pyradiomics}, is built precisely for this
task. 

It is not obvious which of these features will best correlate with the response to
treatment. Even worse, some of them might be highly dependent on the segmentaion of the
ROI, so that if a different professional performs segments the ROI with slight variations,
the value of the features changes wildly. It is desirable to select those featues which,
first, are stable under variations in segmentation, called \emph{reproducible}, and are
best correlated to the response to treatment. In \cite{features}, a study part of
TOPiomics, this exact task is performed with lung tumour scans from the database at VHIO.
The data from this study has been used as test data to test the library.
\end{document}
