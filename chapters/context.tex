\documentclass[../main.tex]{subfiles}

\begin{document}
\section{The problem}
The data that has been used for the analysis comes from a study in the field of radiomics.
The aim of radiomics is to establish a relationship between a patient's response to
treatment and what are known as \emph{radiomic features} of a lesion. These are parameters
which give a systematic description of a feature gathered from scans. The hypothesis is
that there is information that can be extracted from this more systematic study of medical
scans, as opposed to the more traditional qualitative observation by a medical
professional. 

On a practical level, the radiomic features are extracted from a 3d reconstruction of the
lesion, built from a series of 2d sectional scans. It is then the work of a medical
professional to manually delimit the contours of the actual lesion, which is known as the
\emph{region of interest}. This process is called \emph{segmentation}. Once this is done,
the actual work of computing the radiomic features takes place. For the data at hand,
these had been extracted with the \textsf{PyRadiomics} Python package \cite{pyradiomics}.
These features include shape parameters of the triangular mesh determined by the segmented
region of interest, as well as statistical descriptors of the distribution of pixel
intensity in the region of interest. 





\end{document}
