\documentclass[12pt]{article}

\usepackage[catalan]{babel}
\usepackage[utf8]{inputenc}
\usepackage[T1]{fontenc}
\usepackage{lmodern}

\title{Homologia Persistent}
\author{Arnau Mas}
\date{2020}

\begin{document}
\maketitle

\section{Context de les dades}
Els metges prenen dades en dos dominis diferents: per una banda el \emph{domini radiòmic},
que consisteix en totes les observacions que es poden fer del tumor: eixos major i menor,
elongació, etc., i d'altra banda el \emph{domini genòmic}, que consisteix en
característiques extretes de l'ADN del pacient. A partir d'aquests dos conjunts de dades
es vol entendre l'eficàcia d'un tractament concret contra el càncer de pulmó, és a dir
intentar trobar quines són les característiques radiòmiques o genòmiques que influeixen en
l'èxit o no del tractament. 

\section{Explicació de les eines matemàtiques}
Desenvolupar els següents punts per posar el lector en context
\begin{itemize}
	\item Topologia algebraica, complexos de cadenes, diferencial, cicles i vores, grups
		d'homologia, números de Betti, homologia simplicial, homotopia. 
	\item Homologia persistent, filtracions, grups d'homologia persistent, diagrames de
		persistència.
	\item Eines específiques per a la filtració que es fa servir: graf de \emph{mutual \( k
		\)-nearest neighbours}, complexos de Cech i Viettoris-Rips i la seva relació,
		construcció de la filtració a partir del graf de M\( k \)NN. 
\end{itemize}

\section{Justificació matemàtica del codi}
Per una banda es podria justificar la construcció de la matriu d'adjacència del graf de
M\( k \)NN, que fa servir la matriu de distàncies del núvol de punts.

EL punt més rellevant és justificar el càlcul de l'homologia persistent. 

\section{Discussió dels resultats del programa}
Això queda subjecte a que acabi d'enllestir el programa i l'executi amb les dades. Per
aquesta aplicació en concret hi ha dos paràmetres que interessen per a entendre quan és
que un conjunt de punts és extrany o no: per una banda el seu temps de persistència, però
també el número de punts que acumula (les classes extranyes són les que persisteixen
molt i tenen pocs punts). Es podrien representar les classes (almenys pel cas de \( H^0
\)) en un pla on un eix fos el nombre de punts que tenia en el moment en el que desapareix
i l'altre el seu temps de vida. Pels grups d'homologia d'ordre superior no queda tan clar
què vol dir el "número de punts" que conté una classe. 

\end{document}

