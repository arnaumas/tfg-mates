\documentclass[12pt, oneside]{book}
% -------------------
% PACKAGES
% Basic font setup
\usepackage[english]{babel}
\usepackage[utf8]{inputenc}
\usepackage[T1]{fontenc}
\usepackage{lmodern}

% Figures
\usepackage{graphicx}
\usepackage[font={footnotesize, sf}, labelfont=bf]{caption} 
\graphicspath{{figs/}{../figs/}}

% Maths tools
\usepackage{amsmath, amssymb, mathtools}
\usepackage{amsthm, thmtools}
\usepackage{tikz-cd}

% Format
\usepackage{geometry}

% References
\usepackage{hyperref}
\usepackage[english]{cleveref}
\usepackage{biblatex}
\usepackage{csquotes}

% Utilites
\usepackage{enumerate}

% -------------------
% CUSTOMISATION
% Geometry setup
\geometry{
	a4paper,
	right = 2.5cm,
	left = 2.5cm,
	bottom = 3cm,
	top = 3cm
}
\renewcommand{\baselinestretch}{1.3}

% -------------------
% Reference setup
\hypersetup{
	colorlinks,
	linkcolor = {red!50!blue},
	citecolor = {red!50!blue},
	urlcolor = {red!50!blue},
	linktoc = page
}

% -------------------
% Bibliography
\addbibresource{refs.bib}

% -------------------
% Theorem environments
\newcommand{\qedtriangle}{\ensuremath{\triangle}}
\newcommand{\qedtriangledown}{\ensuremath{\bigtriangledown}}

\declaretheoremstyle[spaceabove=6pt, spacebelow=6pt, headfont=\bfseries,
notefont=\normalfont, notebraces={(}{)}, qed=\qedtriangle]{definition}
\declaretheoremstyle[spaceabove=6pt, spacebelow=6pt, headfont=\bfseries,
notefont=\normalfont, notebraces={(}{)}, qed=\qedtriangledown]{example}

\declaretheorem[name=Theorem, refname={theorem,theorems}, Refname={Theorem,Theorem},
numberwithin = chapter ]{theorem}

\declaretheorem[name=Proposition, refname={proposition,propositions},
Refname={Proposition,Propositions}, numberlike=theorem]{proposition}

\declaretheorem[name=Lemma, refname={lemma,lemmas},
Refname={Lemma,Lemmas}, numberlike=theorem]{lemma}

\declaretheorem[name=Definition, style=definition, refname={definition,definitions},
Refname={Definitio,Definitions}, numberlike=theorem]{definition}

\declaretheorem[name=Example, style=example, refname={example,examples},
Refname={Example,Examples}, numberlike=theorem]{example}

\declaretheorem[name=Remark, style=remark, refname={remark,remarks},
Refname={Remark,Remarks}, numberlike=theorem]{remark}

% -------------------
% LAYOUT
\usepackage[bf,sf,small,pagestyles]{titlesec}
\usepackage{titling}

% Pagestyle defintions
\newpagestyle{main}[\sffamily \footnotesize]{
	\sethead*{\ifthesection{{\bfseries \thesection} \sectiontitle}{}}{}{{\bfseries Chapter \thechapter.} \chaptertitle}
	\headrule
	\footrule
	\setfoot*{}{}{\thepage}
}
\renewpagestyle{plain}[\sffamily \footnotesize]{
	\footrule
	\setfoot*{}{}{\thepage}
}
\renewpagestyle{empty}{}

% Format of chapter titles
\titleformat{\chapter}[block]{\sffamily \bfseries \Huge}{\filleft \large Chapter \Huge \thechapter\\}{0pt}{\Huge \titlerule[1pt] \vspace{1ex} \filleft}

% -------------------
% SUBFILES
\usepackage{subfiles}

\newcommand{\set}[1]{\left\{ #1 \right\}}
\newcommand{\abs}[1]{\left\lvert #1 \right\rvert}

\newcommand{\N}{\mathbb{N}}
\newcommand{\R}{\mathbb{R}}
\newcommand{\F}{\mathbb{F}}
\renewcommand{\S}{\mathfrak{S}}

\makeatletter
\newcommand*{\defeq}{\mathrel{\rlap{%
    \raisebox{0.3ex}{$\m@th\cdot$}}%
  \raisebox{-0.3ex}{$\m@th\cdot$}}%
	=
}
\makeatother

\DeclareMathOperator{\im}{im}

% Category theory
\newcommand{\longto}{\longrightarrow}
\newcommand{\into}{\hookrightarrow}

\newcommand{\MKNN}{M\( k \)NN }
\newcommand{\hombound}[2]{\bar{\partial}_{#1}^{#2}}



\title{}
\author{Arnau Mas}
\date{}

\begin{document}
\begin{titlepage}
	\setstretch{2.0}
	\centering \sffamily

	\vspace*{2cm}

	\includegraphics[width = 8cm]{logo-uab}

	\vspace{2cm}

	{\Large \itshape Mathematics BSc. Undergraduate Thesis} \\
	{\large \itshape July 2020}

	\vspace{10pt}
	\hrule
	\vspace{10pt}
	{\bfseries \LARGE Topological data analysis and persistent homology}

	{\Large Developing an outlier detector based on persistent homology}
	\vspace{10pt}
	\hrule		
	\vspace{2cm}

	{\LARGE Arnau Mas Dorca}

	\vspace{1cm}
	{\large \itshape Supervised by}

	{\Large Dr Albert Ruiz}
\end{titlepage}
\thispagestyle{empty}

\pagestyle{plain}
\frontmatter
{\footnotesize \sffamily \tableofcontents}

\pagebreak
\addcontentsline{toc}{chapter}{Preface}
\section*{Preface}
The present work is the result of an internship in the Interactive Augmented Modelling
(IAM) at the Computer Vision Center (CVC). Since the month of october I have been
contributing to the project TOPiomics, which is an ongoing collaboration,
between the IAM and the Radiomics Group at the Vall d'Hebron Institute of
Oncology (VHIO). TOPiomics aims to use tools from mathematics to aid in the data analysis
aspects of the medical field of radiomics. The usual tools such as machine learning or
even less sophisticated statistical analysis fail because the volume of test data is
limited ---this is a common occurence with clinical data---. One of the possible solutions
to this problem is in the field of \emph{topological data analysis}, which seeks to
apply ideas from topology to answer questions related to the shape and local structure
of the data. 

One of the principal problems that TOPiomics is trying to solve is that of \emph{early outlier
detection}. This has to do with detecting patients, typically cancer patients, who will
potentially not respond well to existing treatments because of abnormalities presented by
their tumour. The idea is then to develop ways of detecting these outliers in the radiomic
feature space, i.e. by analysing the \emph{radiomic signature} of the lesion which is
extracted through imaging techniques.

A first approach was developed in \cite{topiomics} but it was desirable to make it less
dependent on adequate choices for parameter values. These choices ultimately had to do
with the scales at which the detection was performed. It was then proposed to incorporate
ideas from \emph{persistent homology} to help solve this problem, since this field
deals precisely with the problems related to trying to determine the shape of data in a
way independent of scale, or rather, by factoring in scale as part of the process. 

The principal contribution of the thesis, then, is the development and implementation of
an outlier detector based on these ideas, which builds on the existing approach.
\cref{ch:problem} presents a more detailed description of the particulars of radiomics,
outlier detection and TOPiomics. In \cref{ch:theory} we give an overview of the principal
mathematical objects which are used in persistent homology, which are then put to use in
\cref{ch:implementation}, which contains a description of the implementation of the
detector and the different mathematical considerations that went into it. Finally, in
\cref{ch:results}, the detector is tested with an artificially generated test dataset and with actual
clinical data provided by VHIO. The input of VHIO is needed to determine how relevant all
of this is to the medical side, nevertheless the results seem promising from a purely
mathematical standpoint.  

\paragraph{Acknowledgements}
I would like to thank CVC, and more specifically Dr Debora Gil and Dr Oriol Terrades for
agreeing to have me as an intern in their group. I am also grateful to Dr Albert Ruiz
for his valuable and helpful supervision. Finally, I want to thank my family and my father
in particular for their support during the whole process. 

\mainmatter
\pagestyle{main}

\chapter{The problem}\label{ch:problem}
\subfile{chapters/chapter-1}

\chapter{An overview of homology theory}\label{ch:theory}
\subfile{chapters/chapter-2}

\chapter{The implementation of the detector}\label{ch:implementation}
\subfile{chapters/chapter-3}

\chapter{The results}\label{ch:results}
\subfile{chapters/chapter-4}

\backmatter
\pagestyle{plain}
\printbibliography

\end{document}
